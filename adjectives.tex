%------------------------------------------------------------------
% Dutch notes
%------------------------------------------------------------------
\documentclass[letterpaper,11pt]{article}
%---------------------
% Packages
%---------------------
\usepackage{color}
\usepackage[margin=1.0in]{geometry}
\usepackage{tabularx}
\usepackage{scrextend}
\usepackage{graphicx}
%\usepackage{enumitem}
%---------------------

%---------------------
% Macros & Misc
%---------------------
\topmargin=-1.0in
\setlength{\textheight}{10in}
\definecolor{resgrey}{gray}{.85}
% \newcommand{\sectionhead}[1]{{\large
%         {\begin{minipage}[t]{\linewidth}{\textit{\textbf{#1
% \vphantom{p\^{e}}}}\end{minipage}}}}}
\topmargin=-1.0in
\setlength{\textheight}{10in}
\pagenumbering{gobble}
\begin{document}
\begin{flushleft}
    \LARGE{\textbf{Dutch Adjectives}}
\end{flushleft}
%---------------------------------
% Section: Introduction
%---------------------------------
\textbf{Introduction}
\par{Adjectives are used to describe nouns or pronouns. For the purposes of
these notes we will split Dutch adjectives into two groups.}
\begin{enumerate}%[topsep=0pt, parsep=0pt, partopsep=0pt, itemsep=0pt]
    \item Attributive, which comes before a noun.
    \item Predicate, which comes after a noun.
\end{enumerate}
\begin{tabular}[t]{l l l}
    \textbf{Singular} & De grote auto   & Een grote auto \\
                      & Het grote huis  & Een grote huis \\
    \textbf{Plural}   & De grote auto's & Grote auto's \\
                      & De grote huizen & Grote huizen \\
\end{tabular}
\\
\par{Only \textit{het} words used in the singular with the indefinte
article \textit{een} or in an otherwise indefinte context. With no other article
or after words such as \textit{elk (each), ieder (every), welk (which), veel
(much), weinig (little), meer (more), minder (less)} are not flexed and thus do
not get the ending \textit{-e}.}
\par{Attributive adjectives after passive pronouns always have the ending
\textit{-e}, in the singular and plural forms.}
\\
\begin{tabular}[t]{l l}
    \textbf{Singular} & \textbf{Plural} \\
    \hline
    De auto           & Het huis        \\
    Mijn grote auto   & Ons grote huis  \\
    \hline
\end{tabular}
\\
\par{Attributive adjectives before indefinite \textit{de-nouns} have the ending
\textit{-e}, but attrrbutive adjectives before indefinte \textit{het-nouns} do
not.}
\\
\begin{tabular}[t]{l l l}
    \textbf{Article Form} & \textbf{Adjective Form} & \textbf{English} \\
    \hline
    De koffie             & Warme koffie            & Warm coffee      \\
    Het water             & Warm watter             & Warm water       \\
    \hline
\end{tabular}
\\
\par{Some adjectives always end in \textit{-en} and never change their form.
Many of these adjectives are used to describe the material of which an object is
made.}
\\
\begin{tabular}[t]{l l l}
    \textbf{Base} & \textbf{Conjugated} & \textbf{English} \\
    \hline
    Metaal        & Metalen             & Metalic          \\
    Zilver        & Zilveren            & Silver           \\
    Glas          & Glazen              & Glass            \\
    Hout          & Houten              & Wooden           \\
    Wol           & Wollen              & Wool             \\
    Brons         & Bronzen             & Bronze           \\
    \hline
\end{tabular}
\\
\begin{small}
    Note: These types of adjectives are only used before the noun.
\end{small}
\\
If we want to used attributive adjectives after a noun we use the
preposition \textit{van (of)}.
\\
\begin{tabular}[t]{l l}
    \textbf{Predicate Form} & \textbf{Attributive Form} \\
    \hline
    Een hoed van papier     & Een papieren hoed         \\
    A hat of paper          & A paper hat               \\
    \hline
\end{tabular}
\\ \\
Past particples of irregular verbs used as adjectives end in \textit{-en}.
\\
\begin{tabular}[t]{l l}
    Gebakken aardappels & Baked potatoes \\
    Berdorven eten      & Spoiled food   \\
\end{tabular}
\\
Regular verbs used as adjectives instead get the ending \textit{-e}.
\\
\begin{tabular}[t]{l l l}
    Koken      & Het gekookte ei & The boiled egg \\
    Verstuiean & Verstuurde past & Sent mail      \\
\end{tabular}
\\
Some adjectives end in \textit{-en} such as \textit{eigen (own),
verlegen (shy), dronken (drunk) open (open), terreden (satisfied), valwassen
(grown up), even (even)}.
\\
\begin{tabular}[t]{l l}
    Dit is een verlegen kind & This is a shy child \\
    Ze gooiden de dronken man het cafe uit. & They threw the drunk out of the
    cafe \\
\end{tabular}
%---------------------------------
% Section: Adjectives of comparison
%---------------------------------
\\
\textbf{Adjectives of comparison}
\par{Comparitive adjectives are those that are used to compare two or more
things. For example a sentence such as: \textit{The dog is bigger than the cat}}
\\  
%---------------------------------
% Section: Superlative Adjectives
%---------------------------------
\textbf{Superlative Adjectives}
\\
Superlitive adjectives are used for comparing one person or thing with
every other ember of their group. For example: \texit{He was the tallest boy in
the class.}
\\
\begin{tabular}[t]{l l l}
    \textbf{Basic} & \textbf{Comparitive \textit{(d)er}} & \textbf{Superlative \textit{st(e)}} \\
    \hline
    Mooi  & Mooier  & Het mooist(e) \\
    Leuk  & Leuker  & Het leukst(e) \\
    Groot & Groter  & Het groost(e) \\
    Duur  & Duurder & Het durrst(e) \\
    \hline
\end{tabular}
\\ \\
\textbf{Predicate \& Attributive adjectives in comparison} \\
Predicate adjectives in the comparative behave just like adjectives in
their basic form.
\\
\begin{small}
    \indent NL: Het huis van Jan is groot \\
    \indent EN: The home of Jan is large  \\
\end{small}
Predicate adjectives in the superlative, however are always breceded by
het. 
\\
\begin{small}
    \indent NL: Dit huis is het groost\textit{e} \\
    \indent EN: This home is the largest. \\
\end{small}
Attributive adjectives in the superlative always have the ending
\texit{-e}, because they are never used with an indefinite meaning.
\\ \\
\textbf{Irregulars}
\\
Some adjectives have irregular forms for the comparitive and superlative.
\\
\begin{tabular}[t]{l l l l}
    \textbf{Comparative} & & & \textbf{Superlative} \\
    \hline
    Goed & Good & Better & Het best\textit{e} \\
    Veel & Much & Meer & Het meest\texit{e} \\
    Weinig & Little, Few & Minder & Minst\textit{e} \\
    Graag & Gladly, Wllingly & Liever & Het lietst\textit{e} \\
    \hline
\end{tabular}
\\ \\
\textbf{Notes on spelling} 
\\
Adjectives ending in \textit{-r} receive an extra \textit{-d} in the
compariative form.
\\
\begin{small}
    \indent \textit{lekker} becomes \textit{lekkerder} \\
\end{small}
When you have a \textit{-s} or \textit{-f} at the end of an adjective changes to
\textit{-z} or \textit{-v} in the comparative.
\\
\begin{small}
    \indent \textit{Boos} becomes \textit{Bozer} \\
    \indent \textit{Lief} becomes \textit{Liever} \\
\end{small}
Double vowels in a closed syllable becomes single in an open syllable.
\\
\begin{small}
    \indent \textit{Groot} becomes \textit{Groter} \\
\end{small}
Single consonants after a short vowel doubles in the comparative.
\\
\begin{small}
    \indent \textit{Plat} becomes \textit{Platter} \\
\end{small}
\end{document}

% \textbf{Adjectives as adverbs} \\
% \par{When an adjective is used as an adverb it remains uninflected, it doesn't
% get an \textit{-e}.}
% \\
% \begin{small} 
%     NL: Peter speelt goed voetbal, maar Johan speelt beter \\
%     EN: Peter plays soccer well, but Johan plays better. \\
% \end{small} 

% \par{Adjectives ending in \textit{-s} form the superlative with \textit{-t}
% rathan than \textit{-st}.}
% \\
% \textit{Boos} becomes Boost
% \\

%------------------------------------------------------------------
% Dutch Notes
%------------------------------------------------------------------
\documentclass[letterpaper,11pt]{article}
%---------------------
% Packages
%---------------------
\usepackage{color}
\usepackage[margin=1.0in]{geometry}
\usepackage{tabularx}
\usepackage{scrextend}
\usepackage{graphicx}
%---------------------

%---------------------
% Macros & Misc
%---------------------
\topmargin=-1.0in
\setlength{\textheight}{10in}

\pagenumbering{gobble}
\begin{document}

\begin{center}
    \LARGE{Dutch Verbs}
\end{center}
\par{In Dutch there are two main tenses, \textit{present} and \textit{past}.
These two tenses can be further divided into eight other tenses.}
\\
\begin{tabular}[t]{l l}
    1: Simple Present  & 5: Future              \\
    2: Simple Past     & 6: Conditional         \\
    3: Present Perfect & 7: Future Perfect      \\
    4: Pluperfect      & 8: Conditional Perfect \\
\end{tabular}
\\
\par{Each of these eight tenses require you to know both the correct conjugation
    and application.}
\\ 
\textbf{\textit{Conjugation}}
\par{In order to conjugate a verb correctly we must first identify the verb stem.
    Typically in Dutch verb infinitives end in the letters \textit{-en}, so to
    get the correct verb stem we just subtract this ending.}
    \par{*Indent* For example if we have the verb \textit{werken (to work)} the correct form of
    the verb stem would be \textit{werk}.}
\\ 
There are a few rules you must keep in mind when working with verb stems.
\begin{enumerate} %[topsep=0pt, parset=2pt]
    \item Long vowel infinitives require long vowel stems.
        \subitem Example *Tighten row spacing*
    \item A verb stem never ends in two identical consonants
        \subitem Example *Tighten row spacing*
    \item A verb stem never ends in \textit{-v} or \textit{-z}, verbs with these
        endings instead are replaced with \textit{-f} or \textit{-s} respectively.
        \subitem Example *Tighten row spacing*
    \item The stem of an \textit{-ien} verb ends in \textit{-ie}.
        \subitem Example *Tighten row spacing*
\end{enumerate}
\\ \\
\textbf{\textit{Simple Present}}
\par{The \textit{simple present} tense is used to refer to an event that takes place
right now, or the current state of an object.}
\\ 
\begin{tabular}[t]{l l}
    \textbf{Conjugation} \\
    Ik + [stem] + t & We + \textit{infinitive} \\
    Je + [stem] + t & Jullie + \textit{infinitive} \\
    Hij + [stem] + t & Ze + \textit{infinitive} \\
\end{tabular}
\\
In Dutch the simple present is used in four cases.
\begin{enumerate}
    \item To refer to a momentary action that coincides wit the moment we are
        talking about it.
        \subitem Example *Tighten row spacing*
    \item To refer to an ongoing, habitual, or repetitive action or state.
        \subitem Example *Tighten row spacing*
    \item To refer to a future event (in combination with an adverb of time).
        \subitem Example *Tighten row spacing*
    \item To refer to a hypothetical \textit{if-then} situation.
        \subitem Example *Tighten row spacing*
\end{enumerate}
\\ \\
\textbf{\textit{Simple Past}}
\par{Before we go get into the simples past we must make a distinction between two
types of verbs \textit{t-verbs} which have a crude stem ending of
\textit{t,h,f,c,k,s,p} and \textit{d-verbs}. A good way to remember this rule is
with the phrase \textit{pocket fish} which contains all consonant endings that
classify a \textit{t-verb}.}
\\ 
\begin{tabular}[t]{l l}
    \textbf{Conjugation t-verb} \\
    Ik + [stem] + te &  We + [stem] + ten \\
    Je + [stem] + te &  Jullie + [stem] + ten \\
    Hij + [stem] + te & Ze + [stem] + ten \\
\end{tabular}
% These two charts should be on the same row
% Chart titles should be italic, and charts should be striped
\begin{tabular}[t]{l l}
    \textbf{Conjugation d-verb} \\
    Ik + [stem] + de &  We + [stem] + den \\
    Je + [stem] + de &  Jullie + [stem] + den \\
    Hij + [stem] + de & Ze + [stem] + den \\
\end{tabular}
\\ \\
We use the \textit{simple past} tense in Dutch when:
\begin{enumerate}
    \item To refer to events that took place int he past and have no relation to
        the present.
    \item To describe what went on during a certain event in the past.
    \item When we introduce a past action or event by \textit{toen (when)}.
\end{enumerate}
\par{Lets examine these rules in more detail.}
Past events that do not have any bearing on the present.
\begin{enumerate}
    \item If the event or action is still relevant to the present time we
        generally use the present perfect. When referring to past events or
        actions, the present perfect is more common that the simple past.
        \subitem Example *Tighten row spacing*
    \item When we refer to an event that took place in the past, all information
        surrounding that event is set in the simple past.
        \subitem Example *Tighten row spacing*
    \item To set the state in the past we usually use the present perfect. All
        actions and events that are followed are in the simple past.
        \subitem Example *Tighten row spacing*
\end{enumerate}
\\ \\ 
Introducing a past action or event by \textit{toen (when)}.
\begin{enumerate}
    \item When we point to a past event by using \textit{toen (when)}, we
        generally use the simple past tense. If we use the perfect tense it all
        it must be the \textit{pluperfect tense}.
\end{enumerate}
\\ \\
*Separate section* \\
\textbf{\textit{The past participle}}
\begin{enumerate}
    \item For the \textit{present tense} we use a past participle. For the
        conjugation of the past participle we must distinguish between
        \textit{t-verbs} and \textit{d-verbs}.
    \item Past participle = GE + [stem] + t/d %Rephrase this.
    \item A past participle never ends in double \textit{-t} or \textit{-d}.
    \item All verbs that begin with the prefixes \textit{be-, er-, her-, ont-,}
        and \textit{ver-} do not get \textit{-ge} before the \textit{past
        particple}.
\end{enumerate}
\\ \\
\begin{tabular}[t]{l l l}
    \textbf{Infinitive} & \textbf{in English} & \textbf{Past Participle} \\
    Verdelen & To distribute & Verdeeld \\
    Geschieden & To happen & Geschied \\
    Betalen & To pay & Betaald \\
    Ontdekken & To discover & Ontdekt \\
    Erkennen & To acknowledge & Erkend \\
    Herkenner & To recognize & Herkend \\
\end{tabular}
\\ \\
In the category of \textit{her-} verbs, there are a few exceptions.
\begin{tabular}[t]{l l l}
    \textbf{Infinitive} & \textbf{in English} & \textbf{Past Participle} \\
    Herbergen & To accommodate & Geherberd \\
    Herhuisvesten & To relocate & Geherhuisvest \\
    Herstructuren & To restructure & Geherstructureerd \\
\end{tabular}
\\ \\ 
\textbf{\textit{Simple Past}}
\\ \\ 
For the \textit{perfect tense}, we generally use the verb \textit{hebben (to
have)}, and \textit{zijn (to be)}.

\begin{tabular}[t]{l l}
    \textbf{Conjugation} \\
    Ik + heb/ben + p.p. &  We hebben/zijn + p.p. \\
    Je + hebt/bent + p.p. & Jullie hebben/zijn + p.p. \\
    Hij heeft/is + p.p. & Ze hebben/zijn + p.p. \\
\end{tabular}
\\ \\
When referring to an event that took place in the past we usually opt for the
\textit{Present pefect}.
\\
For example using \textit{blaffen (to bark)}.
\\
\textit{Pluperfect}
\\ \\
The pluperfect uses the \textit{simple past} tense of the verbs \texit{hebben}
and \textit{zijn}.
\\ 
\begin{tabular}[t]{l l}
    \textbf{Conjugation} \\
    Ik + had/was + p.p. & We + hadden/waren + p.p. \\
    Je + had/was + p.p. & Jullie + hadden/waren + p.p. \\
    Hij + had/was + p.p. & Ze + hadden/waren + p.p. \\
\end{tabular}
\\ \\
Using pluperfect \\
We use the pluperfect to refer to an event that occurred before another past
event. \\
Ex: Toen wij aankwamen, waren de meosto gasten al gearriveerd. \\
We we came, most guest had already arrived. \\
\\ \\
\textit{The simple perfect} \\ \\
For the future tense we use the auxiliary verb \textit{zullen (will)} \\
\begin{tabular}[t]{l l}
    \textbf{Conjugation} \\
    Ik zal + \textit{infinitive} & We zulln + \textit{infinitive} \\
    Je zult/zal + \textit{infinitive} & Jullie zulln + \textit{infinitive} \\
    Hij zal + \textit{infinitive} & Ze zulln + \textit{infinitive} \\
\end{tabular} 
\\ 
%Make a small note bellow chart
*Note: Both zult \& zal are equally correct. \\ 
When to use zullen \\
Zullen + infinitive is more similar to \textit{shall} than to \textit{will}. \\
Express a promise or proposal. \\ 
Stress that something will most certainly happen. \\
Express that an event is likely going to take place. \\
Ex: Ik zal het nooit meev doen! \\
I shall not do it again. \\
\\ \\ 
How do the dutch typically refer to the future? \\ \\
Often in dutch the word \textit{gaan(to go)} is used. \\
We use it to \\
\begin{enumerate}
    \item Express an intended action (but no promise, proposal, or solemn plan).
    \item To say that an event is going to take place, without stressing the
        certainty or mentioning the probability.
\end{enumerate}
\\ 
% Figure out a format for examples
Ik ga vanarand pannenkoeken bakken \\ 
I am going to bake pancakes tonight \\
\\ \\
If the point in time when the event is going to take place is explicitly
mentioned, we often use the simple present. \\
Ik bak vanavond pannenkoeken. \\
English translation here \\
\\ \\ 
\textit{Future Perfect} \\ \\
The future perfect is rather uncommon in Dutch. It is used when you want to say
something with have been completed. \\ 
%Check over this chart and line up the words
\begin{tabular}[t]{l l}
    \textbf{Conjugation} \\
    Ik zal hebben/zijn + p.p. & We zullen hebben/zijn + p.p. \\
    Je zult/zal hebben/zijn + p.p. & Jullie zullen hebben/zijn + p.p. \\
    Hij zal hebben/zijn + p.p. & Ze zullen hebben/zijn + p.p. \\
\end{tabular} 
\\
%This is comparing two future tenses 
Using the future perfect \\
Future perfect: \\ 
Morgen zal ze het allemaal zijn vorgeten. \\
Tomorrow, she will have forgotten all about it. \\
Present perfect \\ 
Morgen is ze het allemaal vegeten. \\
Tomorrow, she has forgoten about it. \\
*The dutch say: \\
\textit{Tomorrow she has forgotten} as opposed to \textit{She will have forgoten}.
\\ \\ 
\textit{Conditional} \\
The \textit{conditional} tense is used to refer to hypothetical situations. The
conditional tense works the same as the future tense, however instead of the
present tense of the irregular verb \textit{zullen} we use the past tense. \\
%Maybe replace the way this is written
%I zou /textit{infinitive}
%Make charts more clear that it is Hij/Zijn
\begin{tabular}[t]{l l}
    \textbf{Conjugation} \\
    Ik zou + \textit{infinitive} & We zouden + \textit{infinitive} \\
    Je zou + \textit{infinitive} & Jullie zouden + \textit{infinitive} \\
    Hij zou + \textit{infinitive} & Ze zouden + \textit{infinitive} \\
\end{tabular} 
\\
Ex: \textit{zeggen (to say) lulsteren (to listen)} \\
Ik zou zeggen \\ % Maybe would workd as a table
English here \\
Ik zou lulsteren \\
English here \\
\\ \\ 
Using the conditional \\ 
We use the conditional tense to refer to hypothetical situations. The most
common form is the conditional: \\ 
If certain criteria are met, then a certain hypothetical situation would be the
case. \\
%Phrasing
Will and would: For more realistic, we can also use the future tense. \\
If certain criteria are met, then a certain hypothetical situation will be the
case. In this case, however the Dutch normally use the \textit{simple present}.
\\
Present Future: Although we use the past tense of the verb \textit{zullen}, we
are referring to hypothetical situations in the present or future. For past
hypothetical situations we use the conditional perfect.  \\
If-then situations \\
For verbs in the if-clause, the English often use the subjunctive (not 'he was'
but 'he were') % Replace with italics
\\
Ex: \\
Dat zou ik neit doen als ik jou was. \\
I would not do that if I were you. \\
Sometimes the if-then structure is not clear \\
We zouden ons maar verelen.\\ 
We would only be bored. \\ % The example online is much more clear
In English \textit{would} can only occur in the \textit{then-clause}. A sentence
like \textit{If I would...then I would} is not considered correct grammar. This
is not the case in Dutch however. \\ \\
Supposed to be situations \\
\textit{Zouden} is also used when we talk about what should be the case
according to our norms, plans, or expectations. \\
Ex: \\
Hij zou vandaag op tijal komen. \\
He was supposed to be on time today. \\
Hij zou toto januari bilijven. \\
He was going to stay until January. \\
Polite Form \\
We also use zouden when we want to sound more polite. To politely express a
wish, we add the adverb \textit{graag (with pleasure,please)}, which give the
equivilent of the Elgnish \textit{I would like}. \\ 
Ex: \\
Ik zou graag een retourje leiden willen. \\
I would like a return ticket to leiden. \\
We also use \textit{zouden} to make a suggestion in the form, \textit{would it
not be better if}. \\
Zou het niet makkelijker zijn als je geweoon een schaar gebruikte. \\
Would it not be easier if you simply used scissors. \\
Should: Dutch does not have a verb for should, instead they use a combination of
two verbs, \textit{zouden moeten (english)}. If we place the infinitive
\textit{moeten(to have to)} after \textit{zouden (english)} we get the English
equivalent of should. \\
Ex: \\
Dat zou je moten weten. \\
You should know that. \\
When the required action is urgent or presented as a clear command, we often
simply use \textit{moeten}. \\ 
Ex: \\
Je moet better apletten. \\
English translation here. \\
Conditional Perfect \\
Used to refer to hypothetical situations in the past. \\
The conjugation is the same as the future perfect tense, with the only
difference being that the verb \textit{zullen} is conjugated in the simple past
tense. \\
\begin{tabular}[t]{l l}
    \textbf{Conjugation} \\
    Ik zou hebben/zijn + p.p. & We zouden hebben/zijn + p.p. \\
    Je zou hebben/zijn + p.p. & Jullie zouden hebben/zijn + p.p. \\
    Hij zou hebben/zijn + p.p. & Ze zouden hebben/zijn + p.p. \\
\end{tabular} 
\\
Ex: \\
Ik zou hebben gedanst. \\
English translation here. \\
\\
Using the conditional perfect \\
We use the conditional perfect to refer to hypothetical situations in the past.  \\
Als je goed had opgelet, zou habben germerkt dat \\
if you had paind attention, you would have noticed that. \\
Using the pluperfect to express a hypothetical situation in the past. \\
In Dutch we often use the \textit{pluperfect} tense to express a hypothetical
situation in the past. \\
Ex:  \\
Ik had data zeker neit gedaan. \\ 
I certainly would not have done that. \\
%-------------------------------------------------------------
% Some of this sounds redundant and needs to be consolidated.
%-------------------------------------------------------------
Should have \\  
For the past participle should have, the dutch say \textit{hadden moeten +
infinitive}. The Dutch construction is int he pluperfect tense which we often
use instead of the conditional perfect. \\
Ik had niet moeten zeggen. \\
I should not have said that. \\
\end{document}


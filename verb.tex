%------------------------------------------------------------------
% Dutch Notes
%------------------------------------------------------------------
\documentclass[letterpaper,11pt]{article}
%---------------------
% Packages
%---------------------
\usepackage{color}
\usepackage[margin=1.0in]{geometry}
\usepackage{tabularx}
\usepackage{scrextend}
\usepackage{graphicx}
%---------------------

%---------------------
% Macros & Misc
%---------------------
\topmargin=-1.0in
\setlength{\textheight}{10in}

\pagenumbering{gobble}
\begin{document}

\begin{center}
    \LARGE{Dutch Verb Tenses}
\end{center}
\par{In Dutch there are two main tenses, \textit{present tense} and \textit{past
        tense}. These two tenses can be further divided into eight other tenses.}
\\
\begin{tabular}[t]{l l}
    1: Simple Present  & 5: Future              \\
    2: Simple Past     & 6: Conditional         \\
    3: Present Perfect & 7: Future Perfect      \\
    4: Pluperfect      & 8: Conditional Perfect \\
\end{tabular}
\\
\par{Each of these eight tenses require you to know both the correct conjugation
    and application.}
\\ \\
\textbf{Conjugation}
\\
\par{In order to conjugate a verb correctly we must first identify the verb stem.
    Typically in Dutch verb infinitives end in the letters \textit{-en}, so to
    get the correct verb stem we just subtract this ending.}
\\ \\
For example given the verb infinitive \textit{Werken} we get \textit{Werk}.
\\ \\ 
There are a few rules you must keep in mind when working with verb stems.
\\ 
\begin{enumerate}
    \item Long vowel infinitives require long vowel stems.
    \item A verb stem never ends in two identical consonants
    \item A verb stem never ends in \textit{-v} or \textit{-z}, verbs with these
        endings instead are replaced with \textit{-f} or \textit{-s} respectively.
    \item The stem of an \textit{-ien} verb ends in \textit{-ie}.
\end{enumerate}
\\ \\
\textit{Simple Present}
\\
Hi
How are you
\end{document}


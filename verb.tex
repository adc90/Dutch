%------------------------------------------------------------------
% Dutch notes
%------------------------------------------------------------------
\documentclass[letterpaper,11pt]{article}
%---------------------
% Packages
%---------------------
\usepackage{color}
\usepackage[margin=1.0in]{geometry}
\usepackage{tabularx}
\usepackage{scrextend}
\usepackage{graphicx}
\usepackage{enumitem}
%---------------------

%---------------------
% Macros & Misc
%---------------------
\topmargin=-1.0in
\setlength{\textheight}{10in}
\definecolor{resgrey}{gray}{.85}
\newcommand{\sectionhead}[1]{{\large
        {\begin{minipage}[t]{\linewidth}{\textit{\textbf{#1
\vphantom{p\^{e}}}}\end{minipage}}}}}
\topmargin=-1.0in
\setlength{\textheight}{10in}

\pagenumbering{gobble}
\begin{document}
%-------------------------------------------------------------------------
% page 1
%-------------------------------------------------------------------------
%------------------------------------------------------------------
% Section: Heading
%------------------------------------------------------------------
\begin{center}
    \large{Dutch Verbs}
\end{center}
\par{In dutch there are two main tenses, \textit{present} and \textit{past}.
    These two tenses can be further divided into eight other tenses. Each of these
eight tenses require you to know both the correct conjugation and application.}
\begin{flushleft}
    \begin{tabular}[t]{l l}
        \hline
        1: Simple Present  & 5: Future              \\
        2: Simple Past     & 6: Conditional         \\
        3: Present Perfect & 7: Future Perfect      \\
        4: Pluperfect      & 8: Conditional Perfect \\
        \hline
    \end{tabular}
\end{flushleft}
\\
%------------------------------------------------------------------
% Section: Conjugation
%------------------------------------------------------------------
\sectionhead{Conjugation}
\par{In order to conjugate a verb correctly we must first identify the verb
    stem. Typically in dutch verb infinitives end in the letters \textit{-en},
    so to get the correct verb stem we just subtract this ending. For example if
    we have the verb \textit{werken (to work)} the correct form of the verb stem
would be \textit{werk}.}
\\
There are a few rules you must keep in mind when working with verb stems.
\begin{enumerate}[topsep=0pt, parsep=0pt, partopsep=0pt, itemsep=0pt]
    \item Long vowel infinitives require long vowel stems.
        \subitem \small{\textit{Maken} would have the stem \textit{Maak} to
        keep its long vowel sound.}
    \item A verb stem never ends in two identical consonants
        \subitem \small{\textit{Pakken} would have the stem
        \textit{Pak}.}
    \item A verb stem never ends in \textit{-v} or \textit{-z}, verbs with these
        endings instead are replaced with \textit{-f} or \textit{-s} respectively.
        \subitem \small{\textit{Leven} would have the stem \textit{Leef}.}
    \item The stem of an \textit{-ien} verb ends in \textit{-ie}.
        \subitem \small{\textit{Skien} would have the stem \textit{Skie}.}
\end{enumerate}
%------------------------------------------------------------------
% Section: Simple Present
%------------------------------------------------------------------
\sectionhead{Simple Present}
\par{The \textit{simple present} tense is used to refer to an event that takes place
right now, or the current state of an object.}
\begin{flushleft}
    \begin{tabular}[t]{l l}
        \textbf{Conjugation} \\
        \hline
        \textit{Ik} + \textit{stem} + \textit{t} & \textit{We} + \textit{infinitive} \\
        \textit{Je} + \textit{stem} + \textit{t} & \textit{Jullie} + \textit{infinitive} \\
        \textit{Hij} + \textit{stem} + \textit{t} & \textit{Ze} + \textit{infinitive} \\
        \hline
    \end{tabular}
\end{flushleft}
\\
In dutch the simple present is used in four cases.
\begin{enumerate}[topsep=0pt, parsep=0pt, partopsep=0pt, itemsep=0pt]
    \item To refer to a momentary action that coincides wit the moment we are
        talking about it.
        \subitem \small{NL: Ik neem een hapje.} 
        \subitem \small{EN: I am taking a bite.} 
    \item To refer to an ongoing, habitual, or repetitive action or state.
        \subitem \small{NL: Ze werkt bij de overheid.} 
        \subitem \small{EN: She works for the government.} 
    \item To refer to a future event in combination with an adverb of time.
        \subitem \small{NL: Ik neem er straks nog een.} 
        \subitem \small{EN: I will have another one in a moment.} 
    \item To refer to a hypothetical \textit{if-then} situation.
        \subitem \small{NL: Als je te cola drinkt krijg je gaatjes in je
        tanden.} 
        \subitem \small{EN: If you drink to much coke you will get
        cavities in your teeth.} 
\end{enumerate}
%-------------------------------------------------------------------------
% page 2
%-------------------------------------------------------------------------
\clearpage
%-------------------------------------------------------------------------

%------------------------------------------------------------------
% Section: Simple Past
%------------------------------------------------------------------
\sectionhead{Simple Past}
\\
\par{Before we go get into the simples past we must make a distinction between
    two types of verbs \textit{t-verbs} which have a crude stem ending of
    \textit{t,h,f,c,k,s,p} and \textit{d-verbs}. A good way to remember this
    rule is with the phrase \textit{pocket fish} which contains all consonant
    endings that classify a \textit{t-verb}.}
\\ \\
\begin{tabular}[t]{l l}
    \textbf{Conjugation t-verb} \\
    \hline
    \textit{Ik} + \textit{stem} + \textit{te} &  \textit{We} + \textit{stem} + \textit{ten} \\
    \textit{Je} + \textit{stem} + \textit{te} &  \textit{Jullie} + \textit{stem} + \textit{ten} \\
    \textit{Hij} + \textit{stem} + \textit{te} & \textit{Ze} + \textit{stem} + \textit{ten} \\
\end{tabular}
\begin{tabular}[t]{l l}
    \textbf{Conjugation d-verb} \\
    \hline
    \textit{Ik} + \textit{stem} + \textit{de} &  \textit{We} + \textit{stem} +
    \textit{den} \\
    \textit{Je} + \textit{stem} + \textit{de} &  \textit{Jullie} + \textit{stem}
    + \textit{den} \\
    \textit{Hij} + \textit{stem} +\textit{de} & \textit{Ze} + \textit{stem} +
    \textit{den} \\
    \hline
\end{tabular}
\\
We use the \textit{simple past} tense in dutch when:
\begin{enumerate}[topsep=0pt, parsep=0pt, partopsep=0pt, itemsep=0pt]
    \item Past events that do not have any bearing on the present
        \begin{itemize}
            \item If the event or action is still relevant to the present time we
                generally use the present perfect. When referring to past events or
                actions, the present perfect is more common that the simple past.
                \subitem \small{NL: Karel de grote regeerde van 800 tot 814.} 
                \subitem \small{EN: Charlemangne reigned from 800 until 814.} 
        \end{itemize}
    \item Describing what went on during a certain past event.
        \begin{itemize}
            \item When we refer to an event that took place in the past, all information
                surrounding that event is set in the simple past.
                \subitem \small{NL: Tijdens de kabinetscrisis was de premier op
                vakantie. } 
                \subitem \small{EN: During the cabinet crisis, the prime-minister was on
                vacation.} 
        \end{itemize}
    \item when we introduce a past action or event by \textit{toen (when)}.
        \begin{itemize}
            \item When we point to a past event by using \textit{toen}, we
                generally use the simple past tense. If we use the perfect tense it all
                it must be the \textit{pluperfect tense}.
                \subitem \small{NL: Toen ik wakker werd, scheen de zon volop.} 
                \subitem \small{EN: When i woke up, the sun was shining brightly.} 
        \end{itemize}
\end{enumerate}
\\ \\
%------------------------------------------------------------------
% section: Past Participle
%------------------------------------------------------------------
\sectionhead{Past Participle}
\begin{enumerate}[topsep=0pt, parsep=0pt, partopsep=0pt, itemsep=0pt]
    \item A past particple never ends in double \textit{d} or \textit{d}.
        \subitem \textit{Rusten} becomes \textit{Gerust}
        \subitem \textit{Bloeden} becomes \textit{Gebloed}
    \item All verbs that begin with the prefixes \textit{be-, er-, her-, ont-,}
        and \textit{ver-} do not get \textit{-ge} before the \textit{past}
        \subitem \textit{Verdelen } becomes \textit{Verdeeld}
        \subitem \textit{Geschieden} becomes \textit{Geschied}
    \item in the category of \textit{her-} verbs, there are a few exceptions.
        \subitem \textit{Herbergen} becomes \textit{Geherberd}
        \subitem \textit{Herhuisvesten} becomes \textit{Geherstructureerd}
\end{enumerate}
\\ \\
%------------------------------------------------------------------
% section: Present Perfect
%------------------------------------------------------------------
\sectionhead{Present Perfect}
\par{For the \textit{present perfect} tense, we generally use the verb
    \textit{hebben (to have)}, and \textit{zijn (to be)}. This can be a bit
    confusing because it is precisely what we said about the \textit{simple
    past}. There are no strict rules to tell you when to use which tense
    however. Generally speaking if you know when to use the \textit{simple past},
you can safely use the \textit{present perfect} in all other cases.}
\\
\begin{tabular}[t]{l l}
    \textit{Conjugation} \\
    \hline
    \textit{Ik} \textit{heb/ben} + \textit{p.p.} &  \textit{We hebben/zijn} +
    \textit{p.p.} \\
    \textit{Je} \textit{hebt/bent} + \textit{p.p.} & \textit{Jullie hebben/zijn}
    + \textit{p.p.} \\
    \textit{Hij} \textit{heeft/is} + \textit{p.p.} & \textit{Ze hebben/zijn} +
    \textit{p.p.} \\
    \hline
\end{tabular}

%-------------------------------------------------------------------------
% Page 3
%-------------------------------------------------------------------------
\clearpage
%-------------------------------------------------------------------------
%------------------------------------------------------------------
% Section: Pluperfect
%------------------------------------------------------------------
\sectionhead{Pluperfect}
\\
\par{The \textit{pluperfect} works in the same way as the \textit{present
    perfect}, except that we use the \textit{simple past} tense of the
verbs \textit{hebben} and \textit{zijn}.} \\
\begin{tabular}[t]{l l}
    \textit{Conjugation} \\
    \hline
    \textit{Ik} + \textit{had/was} + \textit{p.p.} & \textit{We} + \textit{hadden/waren} + \textit{p.p.} \\
    \textit{Je} + \textit{had/was} + \textit{p.p.} & \textit{Jullie} + \textit{hadden/waren} + \textit{p.p.} \\
    \textit{Hij} + \textit{had/was} + \textit{p.p.} & \textit{Ze} + \textit{hadden/waren} + \textit{p.p.} \\
\end{tabular}
\\
\par{We use the pluperfect to refer to an event that occurred before another past
event.} 
\begin{small}
    NL: Toen wij aankwamen, waren de meosto gasten al gearriveerd.
    EN: We we came, most guest had already arrived.
\end{small}
\\ \\
%------------------------------------------------------------------
% Section: Simple future
%------------------------------------------------------------------
\sectionhead{Simple Future}
\par{For the \textit{simple future} tense we use the auxiliary verb
    \textit{zullen (will)}. This tense however is quite a formal way to talk about
the future. Bellow we also show the more common way to refer to the future.} \\
\begin{tabular}[t]{l l}
    \textit{Conjugation} \\
    \hline
    Ik zal + \textit{infinitive} & We zulln + \textit{infinitive} \\
    Je zult/zal + \textit{infinitive} & Jullie zulln + \textit{infinitive} \\
    Hij zal + \textit{infinitive} & Ze zulln + \textit{infinitive} \\
    \hline
\end{tabular} \\
Note: Both zult and zal are equally correct.
\par{The \textit{zullen} + \textit{infinitive} construction is more similar to the English word \textit{shall} than to
\textit{will}. We use this form in the following situations.}
\begin{enumerate}[topsep=0pt, parsep=0pt, partopsep=0pt, itemsep=0pt]
    \item To express a promise or proposal.
        \subitem \small{NL: Ik zal het nooit meer doen!} 
        \subitem \small{EN: I shall not do it again.} 
    \item Stress that something will most certainly happen.
        \subitem \small{NL: Je zult dat nog nodig hebben.} 
        \subitem \small{EN: You are going to need it.} 
    \item Express that an event is likely going to take place.
        \subitem \small{NL: Hij zal het waarschijnlijk morgen bekendmaken.} 
        \subitem \small{EN: He will probably announce it tomorrow.} 
\end{enumerate}
\textit{\textbf{\small{Common Future}}}
\par{As previously mentioned the \textit{simple future} is a more formal
    gramatical construction. Instead the Dutch typically refer to the future using
what we'll call here the \textit{common future}.}
\\
Often in dutch the word \textit{gaan(to go)} is used to:
\begin{enumerate}[topsep=0pt, parsep=0pt, partopsep=0pt, itemsep=0pt]
    \item Express an intended action but no promise, proposal, or solemn plan.
        \subitem \small{NL: Ik ga vanarand pannenkoeken bakken.} 
        \subitem \small{EN: I am going to bake pancakes tonight.} 
    \item To say that an event is going to take place, without stressing the
        certainty or mentioning the probability.
        \subitem \small{NL: Zij gaat morgen een liedje voor ons zigen.} 
        \subitem \small{EN: Tomorrow, she is going to sing a song for us.} 
    \item If the point in time when the event is going to take place is explicitly
        mentioned, we often use the simple present.
        \subitem \small{NL: Ik bak vanavond pannenkoeken.} 
        \subitem \small{EN: Tonight, I will bake pancakes.} 
\end{enumerate}
%------------------------------------------------------------------
% Section: Future Perfect
%------------------------------------------------------------------
\sectionhead{Future Perfect}
\par{The future perfect is rather uncommon in Dutch. It is used when you want to
    say something with have been completed. For the conjugation of the
    \textit{future perfect}, you have to combine your knowledge of the
\textit{future} and \textit{perfect} tense.}
\\
\begin{tabular}[t]{l l}
    \textit{Conjunction} \\
    \hline
    Ik zal hebben/zijn + p.p. & We zullen hebben/zijn + p.p. \\
    Je zult/zal hebben/zijn + p.p. & Jullie zullen hebben/zijn + p.p. \\
    Hij zal hebben/zijn + p.p. & Ze zullen hebben/zijn + p.p. \\
    \hline
\end{tabular}
\\
%This is comparing two future tenses
\par{We use the \textit{future perfect} to say that something will have been
    completed at some point int he future. This tense is quite uncommon in Dutch
    instead they usually opt for the \textit{present perfect}. Bellow we will
compare these two forms and how they're used.}
\begin{enumerate}[topsep=0pt, parsep=0pt, partopsep=0pt, itemsep=0pt]
    \item \textit{Future Perfect - less common}
        \subitem \small{NL: Morgen zal ze het allemaal zijn vergeten.} 
        \subitem \small{EN: Tomorrow, she has forgotten all about it.} 
    \item \textit{Present Perfect - more common}
        \subitem \small{NL: Morgen is ze het allemaal vergeten.} 
        \subitem \small{EN: Tomorrow, she has forgotten all about it.} 
\end{enumerate}
\textit{Tomorrow she has forgotten} as opposed to \textit{She will have forgoten}.
\\ \\
%------------------------------------------------------------------
% Section: Conditional
%------------------------------------------------------------------
\sectionhead{Conditional}
\par{The \textit{conditional} tense is used to refer to hypothetical situations.
    The \textit{conditional} tense works the same as the future tense, however
    instead of the present tense of the irregular verb \textit{zullen} we use the
past tense.}
\\
\begin{tabular}[t]{l l}
    \textit{Conjugation} \\
    \hline
    Ik zou + \textit{infinitive} & We zouden + \textit{infinitive} \\
    Je zou + \textit{infinitive} & Jullie zouden + \textit{infinitive} \\
    Hij zou + \textit{infinitive} & Ze zouden + \textit{infinitive} \\
    \hline
\end{tabular}
\\ \\
%Sub-section: Will and would
\textbf{Will and would}
\par{For more realistic, we can also use the future tense. If certain
    criteria are met, then a certain hypothetical situation will be the case. In
this case, however the Dutch normally use the \textit{simple present}.}
\\
%Sub-section: Present and future
\textbf{Present and future} 
\par{Although we use the past tense of the verb \textit{zullen}, we are
    referring to hypothetical situations in the present or future. For past
hypothetical situations we use the conditional perfect.}
\\
%Sub-section: If-then situations
\textbf{If-then situations}
\par{For the verbs in the \textit{if-clause}, the English often use the
subjunctive (not \textit{he was} but \textit{her were}).} 
\begin{small}
    NL: Dat zou ik niet doen als ik jou was.\\
    EN: I would not do that if I were you.
\end{small}
\par{Sometimes, the \textit{if-then} structure is not clearly visible.}
\begin{small}
    NL: We zouden ons maar vervelen. \textit{(als we er te lang bleven)}
    EN: We would only be bored. \textit{(if we stayed there to long)}
\end{small}
\\
%Sub-section: Supposed to be situations
\textbf{Supposed to be situations}
\par{\textit{Zouden} is also used when we talk about what should be the case
according to our norms, plans, or expectations.} \\
\begin{small}
    NL: Hij zou toto januari bilijven. \\
    EN: He was going to stay until January. \\
\end{small}
%Sub-section: Polite Form
\textbf{Polite Form}
\par{We also use zouden when we want to sound more polite. To politely express a
    wish, we add the adverb \textit{graag (with pleasure,please)}, which give the
equivilent of the English \textit{I would like}.}
\\
\begin{small}
    NL: Ik zou graag een retourje leiden willen. \\
    EN: I would like a return ticket to leiden. \\
\end{small}
\par{We also use \textit{zouden} to make a suggestion in the form, \textit{would
it not be better if}.}
\\
\begin{small}
    NL: Zou het niet makkelijker zijn als je geweoon een schaar gebruikte? \\
    EN: Would it not be easier if you simply used scissors? \\
\end{small}
\textbf{Should}
\par{Dutch does not have a verb for should, instead they use a combination of
    two verbs, \textit{zouden moeten (english)}. If we place the infinitive
    \textit{moeten(to have to)} after \textit{zouden (english)} we get the English
equivalent of should.}
\begin{small}
    NL: Dat zou je moten weten. \\
    EN: You should know that. \\
\end{small}
\par{When the required action is urgent or presented as a clear command, we often
simply use \textit{moeten}.}
\\
\begin{small}
    NL: Je moet better apletten. \\
    EN: English translation here. \\
\end{small}
%------------------------------------------------------------------
% Section: Conditional Perfect
%------------------------------------------------------------------
\sectionhead{Conditional Perfect}
\par{Used to refer to hypothetical situations in the past. \\
    The conjugation is the same as the future perfect tense, with the only
    difference being that the verb \textit{zullen} is conjugated in the simple past
tense.}
\\
\begin{tabular}[t]{l l}
    \textbf{Conjugation} \\
    \hline
    Ik zou hebben/zijn + p.p. & We zouden hebben/zijn + p.p. \\
    Je zou hebben/zijn + p.p. & Jullie zouden hebben/zijn + p.p. \\
    Hij zou hebben/zijn + p.p. & Ze zouden hebben/zijn + p.p. \\
    \hline
\end{tabular}
\\
\par{We use the conditional perfect refers to hypothetical situations in the
    past. Just like the \textit{simple conditional} tense, it mainly servers to
express a conditional clause.}
\\
\begin{small}
    NL: Als je goed had opgelet, zou habben germerkt dat \\
    EN: If you had paind attention, you would have noticed that. \\
\end{small}
\textbf{Using the pluperfect to express a hypothetical situation in the past.}
\par{In Dutch we often use the \textit{pluperfect} tense to express a hypothetical
situation in the past.}
\begin{small}
    NL: Ik had data zeker neit gedaan. \\
    EN: I certainly would not have done that. \\
\end{small}
%-------------------------------------------------------------
% Some of this sounds redundant and needs to be consolidated.
%-------------------------------------------------------------
\textbf{Should have}
\par{For the past participle should have, the dutch say \textit{hadden moeten +
    infinitive}. The Dutch construction is int he pluperfect tense which we often
use instead of the conditional perfect.} 
\\
\begin{small}
     NL: Ik had niet moeten zeggen.   \\
     EN: I should not have said that. \\
\end{small}
\end{document}

% \begin{tabular}[t]{l l l}
%     \textbf{Infinitive} & \textbf{In English} & \textbf{Past Participle} \\
%     \hline
%     Verdelen & To distribute & Verdeeld \\
%     Geschieden & To happen & Geschied \\
%     Betalen & To pay & Betaald \\
%     Ontdekken & To discover & Ontdekt \\
%     Erkennen & To acknowledge & Erkend \\
%     Herkenner & To recognize & Herkend \\
% \end{tabular}

% \begin{tabular}[t]{l l l}
%     \textbf{Infinitive} & \textbf{In English} & \textbf{Past Participle} \\
%     \hline
%     Herbergen & To accommodate & Geherberd \\
%     Herhuisvesten & To relocate & Geherhuisvest \\
%     Herstructuren & To restructure & Geherstructureerd \\
% \end{tabular}

% when referring to an event that took place in the past we usually opt for the
% \textit{Present pefect}.
% \\
% For example using \textit{blaffen (to bark)}.

